% options:
% thesis=B bachelor's thesis
% thesis=M master's thesis
% czech thesis in Czech language
% english thesis in English language
% hidelinks remove colour boxes around hyperlinks

\documentclass[thesis=M,czech]{FITthesis}[2012/06/26]

\usepackage[utf8]{inputenc} % LaTeX source encoded as UTF-8

\usepackage{graphicx} %graphics files inclusion
 \usepackage{amsmath} %advanced maths
 \usepackage{amssymb} %additional math symbols
 \usepackage{color}
% \usepackage{tabularx,colortbl}
% \usepackage[chapter]{algorithm}
 \usepackage{algorithmic}

\usepackage{dirtree} %directory tree visualisation

% % list of acronyms
% \usepackage[acronym,nonumberlist,toc,numberedsection=autolabel]{glossaries}
% \iflanguage{czech}{\renewcommand*{\acronymname}{Seznam pou{\v z}it{\' y}ch zkratek}}{}
% \makeglossaries

\newcommand{\tg}{\mathop{\mathrm{tg}}} %cesky tangens
\newcommand{\cotg}{\mathop{\mathrm{cotg}}} %cesky cotangens
\newcommand{\pozn}[1]{ {\color{red}{#1}} }

% % % % % % % % % % % % % % % % % % % % % % % % % % % % % % 
% ODTUD DAL VSE ZMENTE
% % % % % % % % % % % % % % % % % % % % % % % % % % % % % % 

\department{Katedra teoretické informatiky)}
\title{Řešení soustav lineárnich rovnic v modulární aritmetice na GPU}
\authorGN{Martin} %(křestní) jméno (jména) autora
\authorFN{Černý} %příjmení autora
\authorWithDegrees{Bc. Martin Černý} %jméno autora včetně současných akademických titulů
\supervisor{prof. Ing. Róbert Lórencz, CSc.}
\acknowledgements{Doplňte, máte-li komu a za co děkovat. V~opačném případě úplně odstraňte tento příkaz.}
\abstractCS{V~několika větách shrňte obsah a přínos této práce v~češtině. Po přečtení abstraktu by se čtenář měl mít čtenář dost informací pro rozhodnutí, zda chce Vaši práci číst.}
\abstractEN{Sem doplňte ekvivalent abstraktu Vaší práce v~angličtině.}
\placeForDeclarationOfAuthenticity{V~Praze}
\declarationOfAuthenticityOption{1} %volba Prohlášení
\keywordsCS{Nahraďte seznamem klíčových slov v češtině oddělených čárkou.}
\keywordsEN{Nahraďte seznamem klíčových slov v angličtině oddělených čárkou.}

\begin{document}

% \newacronym{CVUT}{{\v C}VUT}{{\v C}esk{\' e} vysok{\' e} u{\v c}en{\' i} technick{\' e} v Praze}
% \newacronym{FIT}{FIT}{Fakulta informa{\v c}n{\' i}ch technologi{\' i}}

\begin{introduction}
	%sem napište úvod Vaší práce
Inženýrské úlohy mají za úkol určitým způsobem popsat chování reálného fyzikálního světa. Navíc většina inženýrských úloh nelze vyřešit analyticky. Proto je musíme řešit numericky a tímto postupem zkrátka vznikají chyby, ať chceme nebo ne. Tyto chyby vznikají už samotným měřením neboli převodem spojité fyzikální veličiny např. teploty, vzdálenosti.. na diskrétní hodnoty např. stupně, metry.. K řešení inženýrských úloh používáme počítač, což je další zdroj chyb. V počítači neexistují reálná čísla, pouze jejich podmnožiny v podobě čísel s plovoucí desetinou čárkou. Nejvyužívanější takové podmnožiny se označují $float$ resp. $double$ a mají jednoduchou resp. dvojitou přesnost.  Všechny výsledky z reálných čísel se zaokrouhlují na čísla z této podmnožiny.
Nebo můžeme v počítači využívat podmnožinu celých čísel $integer$, což je průnik celých čísel a relativně k mohutnosti celých čísel malého intervalu.
V této práci se budeme snažit počítat soustavy lineárních rovnic tak, aby nevznikali chyby zaokrouhlování během výpočtů. To znamená použít matematickou teorii vybudovanou nad matematickým tělesem, které jsme schopni plně a bez chyby reprezentovat a zpracovávat v počítači. Vhodným nástrojem je modulární resp. více-modulární aritmetika.
Výpočty založených na operacích s maticemi jsou dobře paralelizovatelné a tím se znatelně zrychlí celý výpočet.
K implementaci budu využívat technologii CUDA, což je technologie zahrnující do výpočtu výpočetní výkon GPU.
\end{introduction}

\chapter{Modulární aritmetika}


\section{Generování prvočísel}
%\pozn{na 1xxxxxxxxxxxxxx1 je 3095 prvocisel sieve_m=20bytu sieve_v=2052}
Na více-modulární aritmetiku budu používat jako modula prvočísla. Tím mám zajištěno, že budou mezi sebou nesoudělná. Abych měl co nejméně soustav lineárních kongruencí, budu chtít aby prvočísla byla co největší a zároveň aby se zbytky po dělení vešli do $b$ bitů, viz. . Kvůli násobení při úpravách lineárních kongruencí potřebuji dvounásobek bitů a jeden bit navíc kvůli přetečení při sčítání. CUDA podporuje operace s celými čísly s nejvíce 64bity.
\begin{eqnarray}
2 * b + 1 <= 64 \nonumber \\
b <= 31.5 \nonumber  \\
b = 31 \nonumber
\end{eqnarray}
Ke generování prvočísel použiji Eratosthenovo síto \ref{prg:sito}. Tento algoritmus dokáže určit prvočíselnost u všec celých čísel z intervalu $[0;P] ~ P \in N$. V paměti musím mít u všech takových čísel informaci o prvočíselnosti, stačí jeden bit, který určí zda je nebo není prvočíslo. V případě, že chci určit prvočíselnost až do čísla $2^32-1$, potřebuji 512 MB paměti.
$2^32 bitů = 2^29 bytů = 2^19 kB = 2^9 MB = 512 MB$.
Budou mě zajímat hlavně prvočísla z hlavního intervalu $(2^30;2^31)$, ty mají nejvyšší bit $1$. Těch je polovina tzn. 256 MB. Dále můžeme využít faktu, že sudé číslo nemůže být nikdy prvočíslo. Takže stačí 128 MB pro čísla, která mají nejvyšší a nejnižší bit $1$. K výpočtu budeme muset znát prvočíselnost i u čísel nižších než ta v hlavním intervalu, ale pouze u čísel z pomocného intervalu $(0;\sqrt{2^32})$.
\pozn{TODO: proč stačí $\sqrt{2^32}$}
Algoritmus bude mít dvě části. V první bude klasický algoritmus Eratosthenova síta spuštěný na pomocném intervalu. Ve druhé části budu procházet čísla z hlavního intervalu a zjišťovat zda je dělitelné některým číslem z pomocného intervalu.


\begin{algorithm}[H]
\caption{pseudokód Eratosthenova síta}
\label{prg:sito}
\begin{algorithmic}[1]
\STATE $PRVOCISLA$ = všechna čísla od $2$ do $P$, kandidáti na prvočísla
\DO 
\FOR $PRVOCISLO$ = nejmenší prvočíslo z $PRVOCISLA$
\WHILE $PRVOCISLO$ < $P$
	

\STATE $R$ = všechny řádky matice
\FOR{$krok = 1 \to \lfloor\log_2{N}\rfloor$}\label{prg:pseudoCR_vnejsi}
	\STATE $R$ = každý druhý řádek z $R$
	\FORALL{$radek$ in $R$} \label{prg:pseudoCR_vnitrni}
		\STATE $dist$ = $2^{krok-1}$
		\STATE proveď elementární úpravu řádku $radek$ v závislosti na $dist$
	\ENDFOR
\ENDFOR
\STATE vyřeš rovnici představovanou posledním zbývajícím řádkem
\FOR{$krok = \lfloor\log_2{N}\rfloor \to 1$}\label{prg:pseudoCR_vnejsi2}
	\STATE $R$ = řádky odebrané v první fázi v kroku $krok$
	\FORALL{$radek$ in $R$}\label{prg:pseudoCR_vnitrni2}
		\STATE vypočítej neznámou v matici v řádku $radek$
	\ENDFOR
\ENDFOR
\end{algorithmic}
\end{algorithm}



\chapter{Technologie CUDA}

\chapter{Implementace řešení}

\section{Nastavení Visual Studia}
Pro implementaci a veškerý vývoj používám MS Visual Studio 2010. Nvidia podporuje toto vývojové prostředí a vytváří moduly pro psaní programů v CUDA.
\pozn{vytvoření projektu, souborů .cpp, .cu, instalace CUDA kit...}
Dále musíme projekt nastavit tak, aby dokázal zkompilovat kód v CUDA.


Nyní se pustíme do samotné implementace algoritmů. Základní algoritmus pro řešení SLR je Gaussova eliminace, která převádí matici na horní stupňovitý tvar a poté na jednotkovou matici. My ale použijeme Gauss-Jordanovu eliminaci, která v každém kroku eliminuje podle jednoho řádku - pivotní řádek, a která v každém kroku vytvoří v jednom sloupci samé nuly kromě čísla na diagonále. Gauss-Jordanova eliminace má stejnou složitost \ref jako Gaussova eliminace a navíc se snadněji paralelizuje.

\begin{figure}[h]
\begin{center}
% \includegraphics[width=10cm]{images/GJM.png}
\caption{ Znázornění Gauss-Jordanovy metody }
\label{obr:GJ_slozitost}
\end{center}
\end{figure}



\section{Sekvenční algoritmus}

\subsection{Gauss Jordanův algoritmus}



\begin{conclusion}
	%sem napište závěr Vaší práce
\end{conclusion}

\bibliographystyle{csn690}
\bibliography{mybibliographyfile}

\appendix

\chapter{Seznam použitých zkratek}
% \printglossaries
\begin{description}
	\item[GUI] Graphical user interface
	\item[XML] Extensible markup language
	
	\item[SLR] Soustava lineárních rovnic
	\item[GJM] Gauss-Jordanova metoda
\end{description}


% % % % % % % % % % % % % % % % % % % % % % % % % % % % 
% % Tuto kapitolu z výsledné práce ODSTRAŇTE.
% % % % % % % % % % % % % % % % % % % % % % % % % % % % 
% 
% \chapter{Návod k~použití této šablony}
% 
% Tento dokument slouží jako základ pro napsání závěrečné práce na Fakultě informačních technologií ČVUT v~Praze.
% 
% \section{Výběr základu}
% 
% Vyberte si šablonu podle druhu práce (bakalářská, diplomová), jazyka (čeština, angličtina) a kódování (ASCII, \mbox{UTF-8}, \mbox{ISO-8859-2} neboli latin2 a nebo \mbox{Windows-1250}). 
% 
% V~české variantě naleznete šablony v~souborech pojmenovaných ve formátu práce\_kódování.tex. Typ může být:
% \begin{description}
% 	\item[BP] bakalářská práce,
% 	\item[DP] diplomová (magisterská) práce.
% \end{description}
% Kódování, ve kterém chcete psát, může být:
% \begin{description}
% 	\item[UTF-8] kódování Unicode,
% 	\item[ISO-8859-2] latin2,
% 	\item[Windows-1250] znaková sada 1250 Windows.
% \end{description}
% V~případě nejistoty ohledně kódování doporučujeme následující postup:
% \begin{enumerate}
% 	\item Otevřete šablony pro kódování UTF-8 v~editoru prostého textu, který chcete pro psaní práce použít -- pokud můžete texty s~diakritikou normálně přečíst, použijte tuto šablonu.
% 	\item V~opačném případě postupujte dále podle toho, jaký operační systém používáte:
% 	\begin{itemize}
% 		\item v~případě Windows použijte šablonu pro kódování \mbox{Windows-1250},
% 		\item jinak zkuste použít šablonu pro kódování \mbox{ISO-8859-2}.
% 	\end{itemize}
% \end{enumerate}
% 
% 
% V~anglické variantě jsou šablony pojmenované podle typu práce, možnosti jsou:
% \begin{description}
% 	\item[bachelors] bakalářská práce,
% 	\item[masters] diplomová (magisterská) práce.
% \end{description}
% 
% \section{Použití šablony}
% 
% Šablona je určena pro zpracování systémem \LaTeXe{}. Text je možné psát v~textovém editoru jako prostý text, lze však také využít specializovaný editor pro \LaTeX{}, např. Kile.
% 
% Pro získání tisknutelného výstupu z~takto vytvořeného souboru použijte příkaz \verb|pdflatex|, kterému předáte cestu k~souboru jako parametr. Vhodný editor pro \LaTeX{} toto udělá za Vás. \verb|pdfcslatex| ani \verb|cslatex| \emph{nebudou} s~těmito šablonami fungovat.
% 
% Více informací o~použití systému \LaTeX{} najdete např. v~\cite{wikilatex}.
% 
% \subsection{Typografie}
% 
% Při psaní dodržujte typografické konvence zvoleného jazyka. České \uv{uvozovky} zapisujte použitím příkazu \verb|\uv|, kterému v~parametru předáte text, jenž má být v~uvozovkách. Anglické otevírací uvozovky se v~\LaTeX{}u zadávají jako dva zpětné apostrofy, uzavírací uvozovky jako dva apostrofy. Často chybně uváděný symbol "{} (palce) nemá s~uvozovkami nic společného.
% 
% Dále je třeba zabránit zalomení řádky mezi některými slovy, v~češtině např. za jednopísmennými předložkami a spojkami (vyjma \uv{a}). To docílíte vložením pružné nezalomitelné mezery -- znakem \texttt{\textasciitilde}. V~tomto případě to není třeba dělat ručně, lze použít program \verb|vlna|.
% 
% Více o~typografii viz \cite{kobltypo}.
% 
% \subsection{Obrázky}
% 
% Pro umožnění vkládání obrázků je vhodné použít balíček \verb|graphicx|, samotné vložení se provede příkazem \verb|\includegraphics|. Takto je možné vkládat obrázky ve formátu PDF, PNG a JPEG jestliže používáte pdf\LaTeX{} nebo ve formátu EPS jestliže používáte \LaTeX{}. Doporučujeme preferovat vektorové obrázky před rastrovými (vyjma fotografií).
% 
% \subsubsection{Získání vhodného formátu}
% 
% Pro získání vektorových formátů PDF nebo EPS z~jiných lze použít některý z~vektorových grafických editorů. Pro převod rastrového obrázku na vektorový lze použít rasterizaci, kterou mnohé editory zvládají (např. Inkscape). Pro konverze lze použít též nástroje pro dávkové zpracování běžně dodávané s~\LaTeX{}em, např. \verb|epstopdf|.
% 
% \subsubsection{Plovoucí prostředí}
% 
% Příkazem \verb|\includegraphics| lze obrázky vkládat přímo, doporučujeme však použít plovoucí prostředí, konkrétně \verb|figure|. Například obrázek \ref{fig:float} byl vložen tímto způsobem. Vůbec přitom nevadí, když je obrázek umístěn jinde, než bylo původně zamýšleno -- je tomu tak hlavně kvůli dodržení typografických konvencí. Namísto vynucování konkrétní pozice obrázku doporučujeme používat odkazování z~textu (dvojice příkazů \verb|\label| a \verb|\ref|).
% 
% \begin{figure}\centering
% 	\includegraphics[width=0.5\textwidth, angle=30]{cvut-logo-bw}
% 	\caption[Příklad obrázku]{Ukázkový obrázek v~plovoucím prostředí}\label{fig:float}
% \end{figure}
% 
% \subsubsection{Verze obrázků}
% 
% % Gnuplot BW i barevně
% Může se hodit mít více verzí stejného obrázku, např. pro barevný či černobílý tisk a nebo pro prezentaci. S~pomocí některých nástrojů na generování grafiky je to snadné.
% 
% Máte-li například graf vytvořený v programu Gnuplot, můžete jeho černobílou variantu (viz obr. \ref{fig:gnuplot-bw}) vytvořit parametrem \verb|monochrome dashed| příkazu \verb|set term|. Barevnou variantu (viz obr. \ref{fig:gnuplot-col}) vhodnou na prezentace lze vytvořit parametrem \verb|colour solid|.
% 
% \begin{figure}\centering
% 	\includegraphics{gnuplot-bw}
% 	\caption{Černobílá varianta obrázku generovaného programem Gnuplot}\label{fig:gnuplot-bw}
% \end{figure}
% 
% \begin{figure}\centering
% 	\includegraphics{gnuplot-col}
% 	\caption{Barevná varianta obrázku generovaného programem Gnuplot}\label{fig:gnuplot-col}
% \end{figure}
% 
% 
% \subsection{Tabulky}
% 
% Tabulky lze zadávat různě, např. v~prostředí \verb|tabular|, avšak pro jejich vkládání platí to samé, co pro obrázky -- použijte plovoucí prostředí, v~tomto případě \verb|table|. Například tabulka \ref{tab:matematika} byla vložena tímto způsobem.
% 
% \begin{table}\centering
% 	\caption[Příklad tabulky]{Zadávání matematiky}\label{tab:matematika}
% 	\begin{tabular}{|l|l|c|c|}\hline
% 		Typ		& Prostředí		& \LaTeX{}ovská zkratka	& \TeX{}ovská zkratka	\tabularnewline \hline \hline
% 		Text		& \verb|math|		& \verb|\(...\)|	& \verb|$...$|		\tabularnewline \hline
% 		Displayed	& \verb|displaymath|	& \verb|\[...\]|	& \verb|$$...$$|	\tabularnewline \hline
% 	\end{tabular}
% \end{table}
% 
% % % % % % % % % % % % % % % % % % % % % % % % % % % % 

\chapter{Obsah přiloženého CD}

%upravte podle skutecnosti

\begin{figure}
	\dirtree{%
		.1 readme.txt\DTcomment{stručný popis obsahu CD}.
		.1 exe\DTcomment{adresář se spustitelnou formou implementace}.
		.1 src.
		.2 impl\DTcomment{zdrojové kódy implementace}.
		.2 thesis\DTcomment{zdrojová forma práce ve formátu \LaTeX{}}.
		.1 text\DTcomment{text práce}.
		.2 thesis.pdf\DTcomment{text práce ve formátu PDF}.
		.2 thesis.ps\DTcomment{text práce ve formátu PS}.
	}
\end{figure}

\end{document}
